\documentclass{beamer}
%
% Choose how your presentation looks.
%
% For more themes, color themes and font themes, see:
% http://deic.uab.es/~iblanes/beamer_gallery/index_by_theme.html
%
\mode<presentation>
{
  \usetheme{default}      % or try Darmstadt, Madrid, Warsaw, ...
  \usecolortheme{default} % or try albatross, beaver, crane, ...
  \usefonttheme{default}  % or try serif, structurebold, ...
  \setbeamertemplate{navigation symbols}{}
  \setbeamertemplate{caption}[numbered]
} 

% Packages
\usepackage[english]{babel}
\usepackage[latin1]{inputenc}
\usepackage[T1]{fontenc}
\usepackage{qrcode}

% Author
\title{Very quick introduction to Partial Differential Equations}
\author{Pablo Rodriguez-Sanchez}
\institute{Wageningen University and Research}
\date{\today}

\begin{document}

% Title
\begin{frame}
  \titlepage
  
  \begin{center}
  \qrcode{https://pabrod.github.io}
  \end{center}
  
  \begin{center}
  pabrod.github.io
  \end{center}
  
\end{frame}

% Disclaimer
\begin{frame}{Disclaimer}

These slides are a support for the crash course \textit{Very quick introduction to partial differential equations}, given at Wageningen University in June 2018. As a direct consequence, they will perform poorly if used as a self-study material.

\begin{flushright}
Pablo Rodriguez-Sanchez
\end{flushright}

\end{frame}

% Outline
\begin{frame}{Outline}
 \tableofcontents
\end{frame}

% ----------------------------------------------------------
\section{What's the difference between EDOs and PDEs?}

  \begin{frame}{What's the difference between EDOs and PDEs?}

    How are those two equations different?

    \begin{center}
    $\frac{du}{dx} = -u$
    \end{center}

    \begin{center}
    $\frac{\partial u}{\partial x} = -u$
    \end{center}

  \end{frame}

  \begin{frame}{Exercises}

    Find a function $u(x,y)$ so 

    \begin{displaymath}
    \frac{\partial u}{\partial x} =  x
    \end{displaymath}

    \pause

    and

    \begin{displaymath}
    u(0,y) = \sin y
    \end{displaymath}

    \pause

    \textbf{Solution}:

    \begin{displaymath}
    u(x,y) = \frac{x^2}{2} + \sin y
    \end{displaymath}

  \end{frame}

% ----------------------------------------------------------
\section{Introduction}

  \begin{frame}{Key ideas}

    \begin{itemize}
    \item Derivatives destroy information
        \pause
        \begin{itemize}
        \item That's why ODE problems need initial conditions to be well posed
        \end{itemize}
    \pause
    \item Partial derivatives are weapons of math destruction!
        \pause
        \begin{itemize}
        \item Problem posing with PDEs is hard
        \item PDEs are way more powerful
        \end{itemize}
    \end{itemize}

  \end{frame}

% ----------------------------------------------------------
\section{Vector calculus in a nutshell}

  \begin{frame}{Derivatives in higher dimensions}

    The \textit{nabla} operator for 2-dimensions is defined as:

    \begin{displaymath}
    \vec \nabla = \left( \frac{\partial}{\partial x}, \frac{\partial}{\partial y} \right) 
    \end{displaymath}

    \pause

    When applied to a \textit{scalar} function, it generates the gradient:

    \begin{displaymath}
    \vec {\nabla u}  = \left( \frac{\partial u}{\partial x}, \frac{\partial u}{\partial y} \right) = grad \ u
    \end{displaymath}

    \pause

    When applied to a vector function, it generates the divergence:

    \begin{displaymath}
    \vec \nabla \cdot \vec F = \frac{\partial F_x}{\partial x} + \frac{\partial F_y}{\partial y} = div \ F
    \end{displaymath}

    \pause

    When applied twice, it generates the laplacian:

    \begin{displaymath}
    \vec \nabla \cdot \vec {\nabla u}  = \nabla^2 u = \frac{\partial^2 u}{\partial x^2} + \frac{\partial^2 u}{\partial y^2}
    \end{displaymath}

  \end{frame}

% ----------------------------------------------------------
\subsection{Graphical interpretation}

  \begin{frame}{Graphical interpretation of the gradient}

    \begin{figure}
    \includegraphics[scale=0.4]{img/gradient.png}
    \caption{\label{fig:gradient}Gradient points to the direction of maximum slope}
    \end{figure}

  \end{frame}

  \begin{frame}{Graphical interpretation of the divergence}

    \begin{figure}
    \includegraphics[scale=0.5]{img/divergence.png}
    \caption{\label{fig:divergence}Divergence is maximum in sources and minimum in sinks}
    \end{figure}

  \end{frame}

% ----------------------------------------------------------
\subsection{Summary}

  \begin{frame}{Summary of $\vec \nabla$}

    \begin{table}
    \centering
    \begin{tabular}{l|c|r}
    Operator & Shorthand & Explicit \\\hline
    Gradient & $\vec{\nabla u}$ & $\left( \frac{\partial u}{\partial x}, \frac{\partial u}{\partial y} \right) $\\
    Divergence & $\vec \nabla \cdot \vec F$ & $\frac{\partial F_x}{\partial x} + \frac{\partial F_y}{\partial y}$\\
    Laplacian & $\nabla^2 u$ & $\frac{\partial^2 u}{\partial x^2} + \frac{\partial^2 u}{\partial y^2}$
    \end{tabular}
    \caption{\label{tab:widgets}Most common generalizations of derivatives to higher dimensions.}
    \end{table}

  \end{frame}

% ----------------------------------------------------------
\section{Classical PDEs}

  \begin{frame}{Some classical PDEs}

    All classical PDEs follow the structure:

    \begin{displaymath}
    d \frac{\partial^n u}{\partial t^n} - \vec \nabla \cdot (c \vec{\nabla u}) + a u= f
    \end{displaymath}

    \pause

    \begin{table}
    \centering
    \begin{tabular}{l|c|r}
    n & Type & Some applications \\\hline
    0 & Elliptic & Electrostatics, optimization, fluid dynamics\\
    1 & Parabolic & Heat and chemical diffusion, quantum mechanics\\
    2 & Hyperbolic & Wave motion, electrodynamics
    \end{tabular}
    \caption{\label{tab:classicalPDEs}Examples of the classical PDEs.}
    \end{table}
    
  \end{frame}

  \begin{frame}{Boundary conditions}

    In 1-dimension, boundaries just need a beginning and an end

    \begin{figure}
    \includegraphics[scale=0.7]{img/1d-region.png}
    \caption{\label{fig:1dBoundary}Our region is delimited between A and B.}
    \end{figure}

  \end{frame}

  \begin{frame}{Boundary conditions}

    In more dimensions, boundaries have a shape

    \begin{figure}
    \includegraphics[scale=0.5]{img/2d-region.png}
    \caption{\label{fig:2dBoundary}Our region is... well... it's complicated.}
    \end{figure}

  \end{frame}

  \begin{frame}{Boundary conditions}

    Classical boundary conditions:

    \begin{itemize}
    \item \textbf{Dirichlet}: $u = 0$ in the border
    \pause
    \item \textbf{Von Neumann}: directional derivative in the local perpendicular of the border is zero ($\vec{\nabla u} \cdot \vec n = 0$)
    \pause
    \item \textbf{Periodic}: $u$ is equal at equivalent sides (requires defining what \textit{equivalent sides} means)
    \end{itemize}

  \end{frame}

% ----------------------------------------------------------
\section{Matlab tools}

  \begin{frame}{Matlab tools}
  
    \begin{table}
    \centering
    \begin{tabular}{l|c|r}
    Tool & Dimensions & Form \\\hline
    pdepe & u(x,t) & $c \frac{\partial u}{\partial t} = x^{-m} \frac{\partial}{\partial x}(x^m f) + s$\\
    PDE Toolbox & u(x,y,t) & $d \frac{\partial^n u}{\partial t^n} - \vec \nabla \cdot (c \vec{\nabla u}) + a u= f$
    \end{tabular}
    \caption{\label{tab:MatlabTools}PDE problems and Matlab tool.}
    \end{table}
  
  \end{frame}

\end{document}
